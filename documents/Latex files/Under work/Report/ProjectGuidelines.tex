\documentclass[12pt,a4paper]{article}
\usepackage{graphicx}
\usepackage{biblatex}
\usepackage{parskip}
\usepackage{listings}


\lstset{%
basicstyle=\ttfamily,
breaklines = true,
tabsize=2
}
\graphicspath{ {./Images/} }
\addbibresource{biblography.bib}

\setlength{\parskip}{1em}
\begin{document}
\begin{titlepage}
	\clearpage\thispagestyle{empty}
	\centering
	\vspace{1cm}

	% Titles
	% Information about the University
	{\normalsize ELEC40006 \\ 
		Electronics Design Project\\
		Circuit Simulation\par}
		\vspace{2cm}
	\centering \includegraphics[width=0.5\textwidth]{Logo.jpg}
	\vspace{2cm}

	{\Huge \textbf{Technical Report}} \\
	%\vspace{1cm}
	%{\large \textbf{xxxxx} \par}
	\vspace{4cm}
	{\normalsize Xin Wang\\ CID: 01751352 EIE 1st Year UG \\ \vspace{0.5cm}
	             FIRST LAST \\ \vspace{0.5cm}
	             FIRST LAST\par} \vspace{0.5cm}
	{\normalsize dd-mm-yyyy \par}
	
	\vfill
	\pagebreak
\end{titlepage}

\tableofcontents
\pagebreak

\begin{abstract}
This report describes the design and implementation of a program that is capable of performing a transient simulation
by calculating the node voltages at each successive instant in time. This program parses the netlist file
into a graph data structure, performs analysis using conductance matrices and outputs the results in a CSV format.

-- How accurate is it?
\par
-- Comaparison to commercial software?
\end{abstract}
\pagebreak

\section{Overview of the report}
This report is the distillation of multiple research documents relating to different components of the program. \par
Section 2 gives an abstract view of the design of the program, breaking the program down into 3 modules.
Section 3 provides a summary of the testing methodologies and a comparison to both handwritten results and results of 
established circuit simulator software. 
Section 4 delves into the further work done and some potential ideas to build on.
Section 5, the last section, summarises the report and discusses our overall experiences with the development of this project.
\par
\textit{Talk about added functions and anything else.}
\pagebreak

\section{Project Specification}
	\subsection{Design Problem}
	Develop a program that is able to read in a file describing a circuit specified by the user, perform 
	transient simulation on that circuit and output the calculated voltages at each instance in time into a file. 
	\subsection{Program Requirements}
	The main program requirements are listed as follows:
	\begin{itemize}
		\item Program must support basic circuit components listed as follows \footnote{Advanced component can be supported
		provided development schedule is not constrained.}:
		\begin{itemize}
			\item Resistors
			\item Ideal Capacitors
			\item Ideal Inductors
		\end{itemize}
		\item Input file must adhere to SPICE netlist formats
		\item The output file must be in Comma Separated Value (.CSV) format.
		\begin{itemize}
			\item Columns of output file represent nodes in the circuit.
			\item Rows of output file represent an instance in the simulation.
		\end{itemize}
	\end{itemize}
	\vfill
	\pagebreak

\section{Project Management}
\pagebreak

	\subsection{Project timeline}

	\pagebreak

	\subsection{Management approach}
	\pagebreak

	\subsection{Project responsibilities breakdown}
	\pagebreak

\section{Preliminary Designs}
	\subsection{Object Orientated Programming}
	The program architecture was designed with Object Orientated Programming (OOP) in mind. 
	The OOP approach allows the program design to be rapidly and cost-effectively altered to accomodate new features
	such as additional circuit components or analysis techniques that the client might request later on in the 
	program's lifecycle. 

	An OOP approach has shown to be slower than preliminary programs designed without OOP to an average of of 5\% to 10\% 
	depending on other optimisations in the program. However, the requirements of the client does not 
	explicitly state a need for extremely fast program execution and a definitive list of circuit components was not
	provided...

	\subsection{Modified Nodal Analysis}

	\subsection{Eigen Matrix Library}
	\pagebreak

\section{Program Design}
\pagebreak
	\subsection{Parser module}
	\pagebreak
	\subsubsection{Features}
	\pagebreak
	\subsubsection{Pseudocode}
	\pagebreak

	\subsection{Analysis module}
	\pagebreak
	\subsubsection{Features}
	\pagebreak
	\subsubsection{Pseudocode}
	\pagebreak

	\subsection{Transient module}
	\pagebreak
		\subsubsection{Features}
		\pagebreak
		\subsubsection{Pseudocode}
		\pagebreak

\section{Testing}
\pagebreak

\section{Optimisations}
\subsection{Sparse Matrices}
\pagebreak

\section{Adding on}
\pagebreak

\section{Appendix}
\pagebreak

\section{References}
\printbibliography[title={References}]
\end{document}